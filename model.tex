\section{Model}
In this paper, we will use the classic sporadic task model~\cite{Mok1983}, where a task can be specified as a 4-parameter tuple: $\tau_i=(C_i,D_i,T_i)$, where $C_i$ denotes the worst-case execution time estimate (WCET), $D_i$ denotes the relative deadline and $T_i$ denotes the minimal time separation between two successive job release (or called period). Each job  should execute up to $C_i$ time units before the deadline, and otherwise the system is regarded as failure.

Each task has an origin priority and a promoted priority. After $\tau_i$ release a new job $J_{i}$, it is scheduled according to its origin priority. However $J_i$ will be scheduled according to its promoted priority after $p_i$ time units from its release time.

Given a task system $\tau=\{\tau_1,\tau_2,\ldots, \tau_n\}$ sorted according to their origin priority in descending order, we first make the following \textbf{assumptions}:
\begin{enumerate}
	\item Each task  $\tau_i$ has a original priority $n+i$ and a promotion priority $i$ (a smaller index implies higher priority).
	\item Each task has a fixed promotion point $p_i$. The concerned job $J_{i,x}$ has its promotion point $P_{i,x}=r_{i,x}+p_i$,  where $r_{i,x}$ denotes the release time of $J_{i,x}$.
\end{enumerate}



In this paper, our schedulability analysis of dual priority scheduling is restricted to the case that satisfies the above assumptions. Note that, our analysis in fact can be easily extended to more general systems, and we may explore in the future.  For simplicity, we will only present the restricted version in this paper.





