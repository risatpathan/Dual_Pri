\section{Dual Priority}
% According to Baruah (2003, Dynamic- and Static-priority Scheduling of Recurring Real-time Tasks, Theorem~3), $\tau_i$ is schedulable on a single processor using static priority if and only if for each absolute deadline of a job  $d_{i,k}$ where $k\in N$, there exists an interval $d_{i,k-1}\leq t'\leq d_{i,k}$ for which the following condition holds:
% \begin{equation}
% dbf(\tau_i,t)+\sum_{\tau_j\in\{hp_i\}}rbf_i(\tau_j,t')\leq t'	
% \end{equation} 

% Here the demand bound function captures the maximum execution demand of $\tau_i$ for a time interval length $t$ if it is to meet all deadlines.
% \begin{equation}
% dbf(\tau_i,t)=\left(\lfloor \frac{t-D_i}{T_i}\rfloor+1\right)\times C_i
% \end{equation} 
% On the other hand, the request bound function  $rbf_i(\tau_j,t')$ denotes the maximum amount of time for which $\tau_j$ could \textbf{deny the processor} to lower priority task $\tau_i$ over some interval length of $t'$.



\begin{theorem}
\label{theorem:1}
$\tau_i$ is schedulable on a single processor using dual priority scheduling if  for each absolute deadline of a job  $d_{i,x}$  where $x\in N$, there exists a $t'$ where $t-D_i\leq t'\leq t~(=d_{i,x})$ for which the following condition holds: 
\begin{equation}
\label{eq:1}
C_i+F_i(\tau,t',t)\leq t'
\end{equation} 
where $F_i(\tau,t')$ denotes the maximum possible execution resource consumed by  the other jobs released by tasks in the system  except $J_{i,x}$  during $[0,t']$
\end{theorem}
\begin{proof}
We can prove the statement that if $\tau_i$ is not schedulable, then Equation~\ref{eq:1} would not hold. Suppose that $\tau_i$ is not schedulable, then there are legal event sequences in which deadline $d_{i,x}$ is missed when $\tau_i$ is assigned with the current priority and promotion point. Let $S'$ denote such a sequence where $J_{i,x}$ misses deadline at the earliest time at $t$ (i.e., $d_{i,x}=t$). 

It must be the cases that during $[0,t]$ some other jobs are executing because otherwise the time interval between the last idle instant to $t$ could also construct such a sequence. Then it must be that during the time interval $[0,t']$, other jobs have consumed an amount of resource more than 
\[
F_i(\tau,t',t)>t'-C_i
\]
which contradicts the Equation~\ref{eq:1}.
\end{proof}

Therefore a system  is schedulable on a single processor using dual priority scheduling  if and only if all the tasks in the system meet the requirement in Theorem~\ref{theorem:1}, and hence we can  derive the following theorem.
\begin{theorem}
A system $\tau$ is schedulable by dual priority scheduling if the following condition holds: $\forall~\tau_i\in \tau:~\forall t\geq D_i:~\exists t'\in[t-D_i,t]$ so that 
\[
C_i+F_i(\tau,t',t)\leq t'
\]
\end{theorem}




Unfortunately, it is very hard to know the exact value of $F_i(\tau,t',t)$, and hence we choose to derive an upper bound by considering  each task separately. Let  $f_i(\tau_j,t',t)$ denotes the maximum possible resource consumed by $\tau_j$ during $[0,t']$ in the scenario when all  other tasks  do not release any jobs (except $J_{i,x}$). In this case, the execution $\tau_j$ is independent of interference from other tasks, and hence
\[
F_i(\tau,t')\leq 	 \sum_{\tau_j\in\tau\setminus \tau_i} f_i(\tau_j,t',t)+\lfloor \frac{t-D_i}{T_i} \rfloor \times C_i
\]
where $\lfloor \frac{t-D_i}{T_i} \rfloor \times C_i$ upper bounds the execution of $\tau_i$ itself before $J_{i,x}$.
.






